\documentclass{article}
\usepackage[utf8]{inputenc}
\usepackage{amsmath}
\usepackage{amssymb}
\usepackage{geometry}
\geometry{a4paper, margin=1in}

\title{DMO Syllabus: Comprehensive Formula Sheet for MCQs}
\author{Generated by AI Assistant}
\date{\today}

\begin{document}
\maketitle
\tableofcontents
\newpage

\section{Arithmetic (Commercial Mathematics)}
\subsection{Interest and Finance}
\begin{itemize}
    \item \textbf{Simple Interest (S.I.):}
    $$
    S.I. = \frac{P \times T \times R}{100}
    $$
    \item \textbf{Amount (A):} $A = P + S.I.$
    \item \textbf{Compound Interest (C.I.):}
    $$
    \mathbf{A = P \left(1 + \frac{R}{100}\right)^T} \quad \text{and} \quad C.I. = A - P
    $$
    \item \textbf{Profit/Loss Percentage:}
    $$
    \text{Profit \%} = \frac{SP - CP}{CP} \times 100
    $$
    \item \textbf{Amount with VAT:}
    $$
    \mathbf{\text{Amount with VAT}} = SP \left(1 + \frac{\text{VAT Rate}}{100}\right)
    $$
\end{itemize}

\section{Mensuration (Area, Surface Area \& Volume)}
\subsection{3D Shapes (Prisms, Pyramids, Cones, Spheres)}
\begin{itemize}
    \item \textbf{Right Prism:} (Base Area $A_b$, Perimeter $P$, Height $h$)
    $$
    \text{L.S.A.} = P h \quad ; \quad \mathbf{\text{V} = A_b \times h}
    $$
    \item \textbf{Pyramid:} (Base Area $A_b$, Base Perimeter $P$, Height $h$, Slant Height $l$)
    $$
    \text{L.S.A.} = \frac{1}{2} P l \quad ; \quad \mathbf{\text{V} = \frac{1}{3} A_b \times h}
    $$
    \item \textbf{Cylinder:} (Radius $r$, Height $h$)
    $$
    \text{C.S.A.} = 2\pi r h \quad ; \quad \mathbf{\text{T.S.A.} = 2\pi r (h + r)} \quad ; \quad \mathbf{\text{V} = \pi r^2 h}
    $$
    \item \textbf{Cone:} (Radius $r$, Height $h$, Slant Height $l = \sqrt{h^2 + r^2}$)
    $$
    \text{C.S.A.} = \pi r l \quad ; \quad \mathbf{\text{V} = \frac{1}{3} \pi r^2 h}
    $$
    \item \textbf{Sphere:} (Radius $r$)
    $$
    \mathbf{\text{S.A.} = 4 \pi r^2} \quad ; \quad \mathbf{\text{V} = \frac{4}{3} \pi r^3}
    $$
\end{itemize}

\section{Matrices, Determinants and Linear Equations}
\begin{itemize}
    \item \textbf{Determinant of $\mathbf{2 \times 2}$ Matrix:} $A=\begin{pmatrix} a & b \\ c & d \end{pmatrix}$
    $$
    \mathbf{\det(A) = ad - bc}
    $$
    \item \textbf{Inverse of $2 \times 2$ Matrix:}
    $$
    \mathbf{A^{-1} = \frac{1}{\det(A)} \begin{pmatrix} d & -b \\ -c & a \end{pmatrix}}
    $$
    \item \textbf{Condition for Unique Solution of $AX=B$:}
    $$
    \mathbf{\det(A) \neq 0}
    $$
\end{itemize}

\section{Algebra}
\begin{itemize}
    \item \textbf{Quadratic Formula:} For $ax^2+bx+c=0$, the roots are:
    $$
    \mathbf{x = \frac{-b \pm \sqrt{b^2 - 4ac}}{2a}}
    $$
    \item \textbf{Discriminant:} $\mathbf{\Delta = b^2 - 4ac}$. $\Delta=0$ implies real and equal roots.
    \item \textbf{Factor Theorem:} If $P(a)=0$, then $(x-a)$ is a factor of $P(x)$.
\end{itemize}

\section{Sequence and Series}
\begin{itemize}
    \item \textbf{Arithmetic Progression (AP):} ($a$ = first term, $d$ = common difference)
    $$
    \mathbf{n^{th} \text{ Term:} \quad a_n = a + (n-1)d}
    $$
    $$
    \mathbf{\text{Sum } S_n = \frac{n}{2} [2a + (n-1)d]}
    $$
    \item \textbf{Geometric Progression (GP):} ($r$ = common ratio)
    $$
    \mathbf{n^{th} \text{ Term:} \quad a_n = a r^{n-1}}
    $$
    $$
    \mathbf{\text{Sum } S_n = \frac{a(r^n - 1)}{r-1}} \quad ; \quad \mathbf{S_{\infty} = \frac{a}{1-r}} \quad (|r|<1)
    $$
\end{itemize}

\section{Complex Numbers}
\begin{itemize}
    \item \textbf{Standard Form:} $z = x + iy$.
    \item \textbf{Conjugate:} $\mathbf{\bar{z} = x - iy}$.
    \item \textbf{Modulus (Magnitude):}
    $$
    \mathbf{|z| = \sqrt{x^2 + y^2}}
    $$
    \item \textbf{Polar Form:} $\mathbf{z = r (\cos \theta + i \sin \theta)}$.
    \item \textbf{De Moivre's Theorem:}
    $$
    \mathbf{(\cos \theta + i \sin \theta)^n = \cos(n\theta) + i \sin(n\theta)}
    $$
\end{itemize}

\section{Vectors}
Let $\vec{a}$ and $\vec{b}$ be two vectors, and $\theta$ be the angle between them.
\begin{itemize}
    \item \textbf{Magnitude:} If $\vec{a} = a_1 \hat{i} + a_2 \hat{j} + a_3 \hat{k}$, then $\mathbf{|\vec{a}| = \sqrt{a_1^2 + a_2^2 + a_3^2}}$.
    \item \textbf{Dot Product (Scalar):}
    $$
    \mathbf{\vec{a} \cdot \vec{b} = |\vec{a}| |\vec{b}| \cos \theta}
    $$
    \item \textbf{Angle between Vectors:}
    $$
    \mathbf{\cos \theta = \frac{\vec{a} \cdot \vec{b}}{|\vec{a}| |\vec{b}|}}
    $$
    \item \textbf{Orthogonality Condition (MCQ):} $\mathbf{\vec{a} \cdot \vec{b} = 0}$
    \item \textbf{Cross Product (Vector):}
    $$
    \mathbf{\vec{a} \times \vec{b} = \begin{vmatrix} \hat{i} & \hat{j} & \hat{k} \\ a_1 & a_2 & a_3 \\ b_1 & b_2 & b_3 \end{vmatrix}}
    $$
\end{itemize}

\section{Permutations and Combinations}
\begin{itemize}
    \item \textbf{Permutation ($n P_r$):} (Order matters)
    $$
    \mathbf{n P_r = \frac{n!}{(n-r)!}}
    $$
    \item \textbf{Combination ($n C_r$):} (Order does NOT matter)
    $$
    \mathbf{n C_r = \frac{n!}{r! (n-r)!}}
    $$
    \item \textbf{Permutation with Repetition:} (For $n$ items, with $n_1, n_2, \dots$ alike)
    $$
    \mathbf{\text{Total arrangements} = \frac{n!}{n_1! n_2! \dots}}
    $$
\end{itemize}

\section{Probability}
\begin{itemize}
    \item \textbf{Basic Probability:} $\mathbf{P(A) = \frac{\text{Favorable Outcomes}}{\text{Total Outcomes}}}$.
    \item \textbf{Addition Theorem (General):}
    $$
    \mathbf{P(A \cup B) = P(A) + P(B) - P(A \cap B)}
    $$
    \item \textbf{Addition Theorem (Mutually Exclusive):} $\mathbf{P(A \cup B) = P(A) + P(B)}$.
    \item \textbf{Multiplication Theorem (Independent):} $\mathbf{P(A \cap B) = P(A) \times P(B)}$.
\end{itemize}

\section{Co-Ordinate Geometry}
\begin{itemize}
    \item \textbf{Distance Formula:}
    $$
    \mathbf{D = \sqrt{(x_2 - x_1)^2 + (y_2 - y_1)^2}}
    $$
    \item \textbf{Section Formula (Internal $m:n$):}
    $$
    \mathbf{R = \left(\frac{m x_2 + n x_1}{m+n}, \frac{m y_2 + n y_1}{m+n}\right)}
    $$
    \item \textbf{Slope of a Line ($m$):} $m = \frac{y_2 - y_1}{x_2 - x_1}$.
    \item \textbf{Line (Point-Slope Form):} $\mathbf{y - y_1 = m(x - x_1)}$.
    \item \textbf{Condition for Perpendicular Lines (MCQ):} $\mathbf{m_1 m_2 = -1}$.
    \item \textbf{General Circle Equation:} (Center $(a, b)$, Radius $r$)
    $$
    \mathbf{(x - a)^2 + (y - b)^2 = r^2}
    $$
\end{itemize}

\section{Trigonometry}
\begin{itemize}
    \item \textbf{Compound Angle (Sine):}
    $$
    \mathbf{\sin(A \pm B) = \sin A \cos B \pm \cos A \sin B}
    $$
    \item \textbf{Double Angle (Sine):} $\mathbf{\sin(2A) = 2 \sin A \cos A}$.
    \item \textbf{Double Angle (Cosine):}
    $$
    \mathbf{\cos(2A) = \cos^2 A - \sin^2 A = 2 \cos^2 A - 1}
    $$
    \item \textbf{Sine Rule (Properties of Triangle):}
    $$
    \mathbf{\frac{a}{\sin A} = \frac{b}{\sin B} = \frac{c}{\sin C} = 2R}
    $$
    \item \textbf{General Solution for $\sin \theta = \sin \alpha$:}
    $$
    \mathbf{\theta = n\pi + (-1)^n \alpha}
    $$
    where $n \in \mathbb{Z}$.
\end{itemize}

\section{Geometry (Theorems)}
\begin{itemize}
    \item \textbf{Pythagorean Theorem:} $\mathbf{a^2 + b^2 = c^2}$ (for right triangles).
    \item \textbf{Similar Triangles (Area Ratio):}
    $$
    \mathbf{\frac{\text{Area}_1}{\text{Area}_2} = \left(\frac{\text{Side}_1}{\text{Side}_2}\right)^2}
    $$
    \item \textbf{Circle (Angle in Semicircle):} The angle is $\mathbf{90^\circ}$.
\end{itemize}

\end{document}
